\chapter*{Conclusiones}
\addcontentsline{toc}{chapter}{Conclusiones}
\chaptermark{Conclusiones}

El objetivo principal de estas prácticas era mejorar la herramienta 3D para visualizar ontología, con el fin de tener una alternativa a los softwares 2D tradicionales y evitar algunas limitaciones que normalmente tienen.
La primera fase estuvo orientada a filtrar las clases de la escena y desarrollar características para encontrar una clase lo más rápido posible. En noviembre, realizamos un estudio con distintos participantes del instituto
para recabar más información y comprobar el progreso actual de nuestro proyecto, y muchos de ellos afirmaron que la herramienta de búsqueda era increíblemente útil, y que el software de gestión ontológica no siempre dispone de una.

En el desarrollo de RV, cada herramienta tiene que probarse y analizarse desde diferentes perspectivas, ya que la mayor parte del tiempo nos ocupamos de la experiencia del usuario. Tenemos que implementar las cosas de forma que sean intuitivas para el usuario
y eso es algo que olvidamos más a menudo que nunca. Teníamos un miembro del equipo que se centraba exclusivamente en los prefabricados de interfaz de usuario, y teníamos que mantener actualizaciones constantes en nuestras ramas de trabajo con las versiones más recientes de los elementos de menú o interfaz de la escena.

También tuvimos que prestar mucha atención a la optimización para que la escena se ejecutara con la mayor fluidez posible y no rompiera la sensación de inmersión del usuario. Las simulaciones de RV dependen más del nivel de interacción entre el usuario y la escena que del realismo,
como mucha gente puede creer, así que todo tiene que funcionar de una manera muy específica para satisfacer al usuario. Se realizan pruebas con frecuencia para intentar encontrar fallos y posibles agujeros en el código.

Trabajar en equipo es todo un reto y exige desarrollar una gran capacidad de comunicación. Ser capaz de expresar lo que hay que hacer o sugerir nuevas ideas depende en gran medida de la forma en que exponemos lo que pensamos.

La documentación es una parte crucial de cualquier proyecto, ya que ayuda a los nuevos miembros del equipo y mantiene un control de versiones limpio. Dedicar tiempo a crear diagramas de clases, hacer capturas de pantalla y aclarar las conexiones de los inspectores entre los objetos del juego es la clave para
progreso claro. Una semana después de incorporar cada característica a la versión principal de la escena, se dedicó a documentar el trabajo realizado para esa cuestión.

\section{Trabajo Futuro}

El proyecto está lejos de estar terminado. Ahora que podemos importar y exportar información de los estándares de Gaia-X, podríamos pensar en añadir herramientas de edición a la escena.
Crear y borrar burbujas sería el primer paso, pero luego también tendríamos que añadir conexiones. Las clases deberían poder trasladarse de una ontología a otra, o incluso podrían crearse otras nuevas.

Como esta simulación forma parte del contexto del proyecto Gaia-X, también podríamos llevar a cabo más pruebas y recoger ideas de en qué sería importante centrarse a continuación. Algunas personas han sugerido añadir una función multijugador, de modo que podamos utilizar la simulación tanto
como una aplicación de coworking o como un tour de presentación. Esto significa que una persona interactúa con las ontologías y el resto observa lo que ocurre.

Se podría añadir otro intérprete para importar ontologías desde un formato de archivo diferente, como turtle. La idea es casi la misma que la que ya está implementada pero podría añadir mucha versatilidad al proyecto.

Como podemos ver, hay muchas cosas que se pueden hacer para seguir mejorando la simulación. La RV es una herramienta muy útil, pero aún no conocemos todo su potencial. Tenemos que seguir investigando, desarrollando y probando para encontrar nuevas formas de explorar la información.