\chapter*{Conclusiones}
\addcontentsline{toc}{chapter}{Conclusiones}
\chaptermark{Conclusiones}

El objetivo principal de estas prácticas era mejorar la herramienta 3D para visualizar ontologías, con el fin de tener una alternativa a los softwares 2D tradicionales y evitar algunas limitaciones que normalmente tienen.

La primera fase de mi contribución al proyecto estuvo orientada a filtrar las clases de la escena y desarrollar features para encontrar una clase lo más rápido posible. El sistema de búsqueda recibió en repetidas ocasiones muy buenas devoluciones de parte de mi supervisor, colegas y miembros del equipo de Gaia-X.
Cada nueva herramienta fue pensada y planificada para encajar lo más fácilmente posible con el resto del proyecto. Así, logré agregar exitosamente una manera de leer archivos y construir una escena completa e interactiva respetando la información original. También conseguí desarrollar una manera de separar gráficamente las diferentes ontologías importadas y añadir niveles de interacción para darle dinamismo al visualizador.

En el desarrollo de Realidad Virtual, cada herramienta tiene que probarse y analizarse desde diferentes perspectivas, ya que la mayor parte del tiempo nos ocupamos de la experiencia del usuario. Tenemos que implementar las cosas de forma que sean intuitivas para el usuario
y eso es algo que olvidamos con cierta frecuencia. Teníamos un miembro del equipo que se centraba exclusivamente en los prefabricados de interfaz de usuario, y teníamos que mantener actualizaciones constantes en nuestras ramas de trabajo con las versiones más recientes de los elementos de menú o interfaz de la escena.

También tuvimos que prestar mucha atención a la optimización para que la escena se ejecutara con la mayor fluidez posible y no rompiera la sensación de inmersión del usuario. Las simulaciones de RV dependen más del nivel de interacción entre el usuario y la escena que del realismo,
como mucha gente puede creer, así que todo tiene que funcionar de una manera muy específica para satisfacer al usuario. Se realizan pruebas con frecuencia para intentar encontrar fallos y posibles agujeros en el código.

Trabajar en equipo es todo un reto y exige desarrollar una gran capacidad de comunicación. Ser capaz de expresar lo que hay que hacer o sugerir nuevas ideas depende en gran medida de la forma en que exponemos lo que pensamos.

Considero que luego de mi paso por el instituto y mi tiempo de trabajo, logré agudizar mi comprensión de las implicaciones de un proyecto de Realidad Virtual, mejorar mis habilidades de programación, pulir mis herramientas de comunicación efectiva y practicar conocimientos adquiridos durante mi carrera en Ingeniería Mecatrónica.

\section{Trabajo Futuro}

El proyecto está lejos de estar terminado. Ahora que es posible importar y exportar información de los estándares de Gaia-X, se podría pensar en añadir herramientas de edición a la escena.
Crear y borrar burbujas es el primer paso, pero luego también habría que añadir conexiones. Las clases deberían poder trasladarse de una ontología a otra, o incluso podrían crearse otras nuevas.

Como esta simulación forma parte del contexto del proyecto Gaia-X, también sería interesante llevar a cabo más pruebas y recoger ideas de en qué sería importante centrarse a continuación. Algunas personas han sugerido añadir una función multijugador, de modo que podamos utilizar la simulación tanto
como una aplicación de coworking o como un tour de presentación. Esto significa que una persona interactúa con las ontologías y el resto observa lo que ocurre.

Se podría añadir otro intérprete para importar ontologías desde un formato de archivo diferente, como turtle. La idea es casi la misma que la que ya está implementada pero podría añadir mucha versatilidad al proyecto.

Como podemos ver, hay muchas cosas que se pueden hacer para seguir mejorando la simulación. La Realidad Virtual es una herramienta muy útil, pero aún no conocemos todo su potencial. Tenemos que seguir investigando, desarrollando y probando para encontrar nuevas formas de explorar la información.